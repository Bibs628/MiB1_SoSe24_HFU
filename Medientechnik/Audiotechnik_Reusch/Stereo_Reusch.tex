\newpage
\subsection{Stereo in der Audiotechnik}
\notebox{mehr zum Thema 3D-Audio, auch Stereo, findet ihr hier in diesem E-Book von \href{https://soundparticles.com/resources/ebooks/3daudio/}{Soundparticles}}


Bei Stereo wird versucht, mit 2 Lautsprechern ein frontales Klangbild zu erzeugen. Wenn wir exakt dasselbe Signal an beide Lautsprecher senden, wird eine so genannte “Phantomschallquelle” zwischen ihnen erzeugt, wodurch wir den Ursprung dieses Audiosignals direkt vor uns wahrnehmen. Wenn wir das Lautstärkenverhältnis zwischen den beiden Lautsprechern ändern, können wir den Sound  zwischen dem linken und rechten Lautsprecher „hin und her bewegen“. Bei Stereo werden die Lautsprecher normalerweise in einem Winkel von etwa 60° Grad aufgestellt.  \textit{- Auszug aus: All you need to know about 3D Audio Seite 1} \cite{3DAudio_St:online}
