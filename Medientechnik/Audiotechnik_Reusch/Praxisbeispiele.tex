\subsection{Praxisbeispiele}

\subsubsection{RedBull Symphonic}
Eine Herausforderung ein Orchester über Hip-Hop durchzukommen. Hip-Hop ist sehr laut und es erfordert entsprechend viele Mikrophone an den Instrumenten (meist an jedem eines) um die Klassik noch neben dem Hip-Hop zu hören. Ein Problem ist es, dass die Klassische Musik nur bis zu einem gewissen Wert geht und halt nicht höher.
\href{https://youtu.be/2s__lEzWqo4?si=bGhVtFCQNC-s5NzL}{Kool Savas | Red Bull Symphonic - Das Konzert in voller Länge}\\
Abmischung vom Orchester und zusammenarbeit mit \href{https://de.linkedin.com/in/oliver-voges-ab151672}{Olli Voges} (Casper, Cool Savage, Dragonforce, ...)

\subsubsection{Joo Kraus}
\href{https://youtu.be/hrD4D0ShFBY?si=8nK3cuIGO_KtiYv7}{JOO KRAUS / OMAR SOSA / CHAMBER ORCHESTRA ARCATA - Light In The Sky - Live 2017} 

\subsubsection{Sould Diamonds}
\href{https://youtu.be/EeIQMoYi3gU?si=_4K6PZCx4HAfOsjd}{Soul Diamonds - Night in Tunisia}

\subsubsection{Iron Maiden}
Betreut in München, Systemarbeit aka Leutsprecherarbeit (Ausrichten und Konfigurieren)

\subsubsection{Gast}
Herr Reusch mischt auch für die Band seines Sohnes die Songs ab. \href{https://open.spotify.com/intl-de/track/65qo8OLrleFH5bUOsDxRUo?si=49a2ccd60e154820}{Fallen - Gast (Spotify)}

\subsubsection{d\&b Audiotechnik in Ludwigsburg}
In Ludwigsburg testet der Herrsteller \href{https://www.dbaudio.com/global/de/}{d\&b Audiotechnik} seine neue Produkte bei einem Open-Air Festival. \href{https://youtu.be/UBUGswe_Grw?si=p8W6tscHZTbGxhpX}{Hier mal ein Beispiel.} Einige Lautpsrecher von d\&b sind auch bei uns in der Hochschule in Verwendung.
\newpage
\subsubsection{Planungskonzept Saarpolygon}
Im Saarland ist gerade eine Open-Air Bühne in Planung und Herr Reusch versucht dort die lautsprecheraufstellung optimal zu planen. Dies ist bei dem Objekt des \href{www.Opernfestspiele-saarpolygon.de}{Saarpolygons} jedoch aufgrund der einzigartigen Form für eine Bühne sehr anspruchsvoll. So soll das Ochester beispielweise sich in dem Polygon befinden,