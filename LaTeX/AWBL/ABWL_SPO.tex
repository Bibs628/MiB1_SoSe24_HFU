\section{SPO}

    Nachdem Studierende das Modul erfolgreich abgeschlossen haben, können sie:
    \begin{itemize}
        \item Wissen / Kenntnisse\\
            Das Mikro- und Makro-Umfeld von Medienbetrieben, und wie sie diese beeinflussen, benennen sowie den Stellenwert der Medienbranche in der Volkswirtschaft und Gesellschaft skizzieren.
            \newline
            Erklären, wie Medienunternehmen aus betriebswirtschaftlicher Sicht grundlegend funktionieren sowie die relevanten regulatorischen Bedingungen für das Medienmanagement kennen.
        \item Verstehen\\
            Verstehen, wie sich Medienbetrieb unterschiedlicher Art finanzieren sowie verstehen, welche Rechtsformen Medienbetriebe haben können.
            \newline
            Verstehen, welche strategischen und operativen Entscheidungen Medienunternehmen treffen müssen sowie welche Managementinstrumente Medienunternehmen benutzen (können).
        \item Anwenden\\
            Darlegen, in welchem volkswirtschaftlichen, politischen und regulatorischen Bezugsrahmen Medienbetriebe agieren. 
            \newline
            Benennen, wie einzelne Medienbetriebe ihren Markt bzw. ihre Branche definieren und wie sie dies in ihren Aktivitäten beeinflusst.
        \newpage
        \item Analyse\\
            Analysieren, wie Angebot und Nachfrage von Mediengütern zusammenspielen und wie dies von Medienbetrieben koordiniert wird sowie Investitionsentscheidungen in Medienbetrieben analysieren.
            \newline
            Analysieren, wie Medienunternehmen organisiert sind sowie welche Auswirkungen regulatorische Bedingungen auf Entscheidungen im Medienmanagement haben.
        \item Synthesis\\
            Allgemeine personalpolitische Maßnahmen auf Medienbetriebe übertragen.
        \item Evaluation\\
            Steuerliche Konsequenzen medienbetrieblicher Entscheidungen grob bewerten.
            \newline
            Medienbetriebliche Entscheidungen aus Sicht des Controllings bewerten.
    \end{itemize}
    
\newpage