\newpage\section{Unternehmensgründung}
    \subsection{Grundlagen}

\begin{itemize}
    \item Aus welchen Motiven Gründen Menschen Unternehmen?Was zeichnet einen guten Entrepeneur aus?
    \begin{itemize}
        \item Unzufriedenheit mit Unternehmen
        \item Eigenständigkeit / kein Chef
        \item mehr Freiheiten als Angestellte
        \item höheres Gehalt
        \item Sinnhaftigkeit
        \item eigene Idee Umsetzten
        \item Ausnutzung Marktlücke / Nische
        \item  geringe Auslastung in einem Bereich
    \end{itemize}
    \item Was zeichnet einen guten Entrepreneur aus?
    \begin{itemize}
        \item Kritikfähig
        \item Kompetenz
        \item Work-a-holic
        \item Auslagerung von Kompetenzen
        \item hartnäckig
        \item ausdauernd
        \item Einfühlungsvermögen
        \item Begeisterungsfähigkeit
        \item Überzeugungskraft
        \item Im Interesse des Unternehmens handeln
        \item Verbesserungsfähig
        \item Fehler eingestehen
        \item kein Arschloch
        \item fair sein / beurteilen
        \item nicht nachtragend
    \end{itemize}\newpage
    \item Was sind die Vor- und Nachteile einer Unternehmensgründung?
    \begin{itemize}
        \item höheres Risiko
        \item Insolvenzrisiko
        \item Zeitintensiv
        \item hoher Verwaltungsaufwand
        \item mehr Freiheiten
        \item möglicherweise keine Kunden
        \item höheres persönliches Risiko
        \item anfänglich geringe Gewinne (wenn überhaupt) 
    \end{itemize}
    \item Schritte einer Gründung
    \begin{itemize}
        \item Kapital
        \item Idee
        \item Investment falls nötig
        \item Angestellte
        \item Prototyp
        \item Rechtsform
        \item Business Plans
        \item Kundengewinnung
    \end{itemize}
    \item Faktoren
    \begin{itemize}
        \item Schritte
        \item siehe Risiken
    \end{itemize}
    \item Gründe zum Scheitern
    \begin{itemize}
        \item Schlechte Planung
        \item Unerwartete Ereignisse
        \item Verkalkulation
        \item zu hohe Steuervorzahlungen
        \item schlechte Markteinschätzung
        \item zu hohe Kosten
        \item Mentale Probleme
        \item zu schnelles Wachstum
        \item falsche Prognosen
        \item externe Einflüsse (Katastrophen, Kriege, ...) 
    \end{itemize}
\end{itemize}
    \subsection{Unternehmensplan}
    \subsection{Die 9 Schritte zur Selbstständigkeit}
        \subsubsection{Entscheidung für die Selbstständigkeit}
            \paragraph*{Motive für die Existenzgründung}
            \begin{itemize}
                \item Innovation
                \item Anerkennung
                \item Rollenverhalten
                \item Selbstverwirklichung
                \item Unabhängigkeit
                \item Wirtschaftlicher Erfolg
            \end{itemize}

            
            \paragraph*{Eigenschaften eines Gründers}
            \begin{itemize}
                \item Flexibilität
                \item Machbarkeitsüberzeugung
                \item Risikofreudigkeit
                \item Soziale Kompetenz
                \item Entschlussfreudigkeit
                \item Problemorientierung
                \item Wachstumsorientierung
                \item Durchhaltevermögen
                \item Unabhängigkeitsstreben
                \item Leistungsmotiv
            \end{itemize}
            \paragraph*{ Auslöser der Gründungsaktivität „Theory of planned behavior“}
                „Unternehmensgründungen sind kein spontanes Ereignis zu einem zufälligen Zeitpunkt, sondern das Ergebnis von situativen und kulturellen Faktoren.“\newpage
                \begin{center}
                Äußere oder innere Lebensumstände\\+\\
                Positive Bewertung der Selbstständigkeit\\+\\
                Persönliche hohe Handlungsbereitschaft\\=\\
                Wahrscheinlichkeit der Unternehmensgründung \end{center}
        \subsubsection{Zusammenstellung eines Teams}
            \paragraph{Gründe für eine Gründung im Team}~\\
            \begin{itemize}
                \item Ausgleich der vorhandenen Schwächen (Persönlichkeit, Kompetenz, Know-how)
                \item Größere Finanzkraft
                \item Geteiltes finanzielles Risiko 
                \item Gegenseitige Sparringspartner bei Ideen- und Entscheidungsfindung
            \end{itemize}
            \paragraph{Aber...}~\\
            \begin{itemize}
                \item Verlust an Autonomie
                \item Aufteilen der Erlöse auf mehrere Gründer anteilig an Beteiligung
            \end{itemize}
            \notebox{Eine durchdachte Zusammenstellung des Gründerteams ist entscheidend!}
            \paragraph{Merkmale einer guten Teamzusammenstellung}~\\
            \begin{itemize}
                \item Teile eine gemeinsamen Vision
                \item Gemeinsame Motivation
                \item Hohe Teamfähigkeit und gegenseitige Unterstützung
                \item Offene und regelmäßige Kommunikation
                \item Komplmentäre Eigenschaften und stärken
                \item Klare Vereinbarungen über Eigenstumsverhältnisse
                \item Klare Vereinbarungen über Rechte und Pflichten
                \item Klare Aufteilung der Zuständigkeiten
            \end{itemize}

            \paragraph{Eine häufige Aufteilung in einem Start-Up ist}
            \begin{itemize}
                \item Einer übernimmt die Entwicklung/Technische Leistung
                \item der andere übernimmt den Vertrieb / die kaufmännische Leitung
            \end{itemize}
        \subsubsection{Geschäftsidee entwickeln}
        \subsubsection{Geschäftsmodell konzipieren}
        \subsubsection{Businessplan aufstellen}
        \subsubsection{Finanzierung}
        \subsubsection{Unternehmen gründen}
        \subsubsection{Angebot vermarkten}
        \subsubsection{Erfolg hinterfragen}

    